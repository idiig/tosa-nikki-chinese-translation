\setchapterpreamble[u]{\margintoc}

\begin{margintable}
\caption{符号一览表(1)}
\small
\begin{tabular}{rl}
\toprule
\bfseries 符号 & \bfseries 语法功能\\\midrule
-: & affix boundaries\\
=: & clitic boundaries\\
+: & compound boundaries\\
\sc{adn}: & adnominal\\
\sc{adv}: & adverbal\\
\sc{ccl}: & conclusive\\
\sc{cjt}: & conjectural\\
\sc{cmp}: & complementizer\\
\sc{con}: & concessive\\
\sc{cop}: & copula\\
\sc{hev}: & hearsay evidential\\
\sc{ifo}: & inclusive focus\\
\sc{scm}: & sequential complementizer\\
\sc{seq}: & sequential\\\bottomrule
\end{tabular}
\end{margintable}

\lingset{textanchor=numleft,
labelanchor=numleft,
labeloffset=.35in,
textoffset=.8in}

\pex[tag=id1] 男もすなる日記といふものを、女もしてみむとてするなり。\footnote{%
小松英雄认为,这篇日记并非女性的模仿,而是作者意图用假名而非汉字书写,巧妙地利用了假名的独特特性。这一解释由小松英雄提出,但尚未被广泛接受。小松英雄认为开头的一句话巧妙地表达了用假名书写的意图,强调了其风格特点。\\%
田边圣子认为,使用假名可以更细腻地表达悲伤,而古典汉语则受限于较为生硬的表达方式。\\%
桥本治指出,纪贯之作为诗人,选择使用假名反映了他作为诗人的背景,因为传统上诗歌是用假名书写的。%
}
\a[label=\textbf{\underline{かな}}] 
をとこもすなるにきといふものを、をんなもしてみむとてするなり。
\a[label=\textbf{\underline{直译}}] 
听说是男性所做的叫日记的东西,但想着女性也做着试试吧,便做了。
\a[label=\textbf{\underline{意译}}] 
虽说日记通常是用男人写的东西,但我想试着用作为女人来写写看,于是便写下了这本日记。
\a
\begingl[glstyle=nlevel,glneveryline={\it,\it\footnotesize,\sc\footnotesize,\bfseries\footnotesize},
glnabovelineskip={,-.4ex,-.4ex},extraglskip=0pt,glwordalign=center]
\glpreamble 男もすなる日記
\endpreamble
男[onoko///男人]
も[=mo/=\sc{ifo}//也]
す[s-u/做-\sc{ccl}//做]
なる[nar-u/\sc{hev}-\sc{adn}/传闻证据/传闻是...的]
日記[nikki///日记]
\glft `传闻是男人做的日记'
\endgl
\a
\begingl[glstyle=nlevel,glneveryline={\it,\it\footnotesize,\sc\footnotesize,\bfseries\footnotesize},
glnabovelineskip={,-.4ex,-.4ex},extraglskip=0pt,glwordalign=center]
\glpreamble というものを \endpreamble
と[=to/=\sc{cmp}/补语/]
いふ[ip-u/说-\sc{adn}/{}/所谓的]
もの[mono/{}/{}/东西]
を[=wo/=\sc{con}/让步/但]
\glft `虽说,但'
\endgl
\a
\begingl[glstyle=nlevel,glneveryline={\it,\it\footnotesize,\sc\footnotesize,\bfseries\footnotesize},
glnabovelineskip={,-.4ex,-.4ex},extraglskip=0pt,glwordalign=center]
\glpreamble 女もしてみむとてするなり \endpreamble
女[womina///女人]
も[=mo/=\sc{ifo}]
し[s-i/做-\sc{adv}//做]
て[=te/=\sc{seq}/接续/着]
み[m-i/试-\sc{adv}//试试]
む[+m-u/+\sc{cjt}-\sc{ccl}/意志/吧]
とて[=tote/=\sc{scm}/补语接续/想着]
する[s-ur-u/做-\sc{adn}//做]
なり[nar-i/\sc{cop}-\sc{ccl}/系辞/乃]
\glft `想着女性也做着试试吧,便做了。'
\endgl
\xe

\newpage
